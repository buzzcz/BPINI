\chapter{Sada ukázkových testů a~jejich scénářů}
Pro potřeby práce a~pro názornou ukázku odlišnosti přístupu k~testování webové a~desktopové aplikace byla vytvořena téměř identická aplikace Převodník pomocí technologie JavaFX. Následující sada testů a~jejich scénářů se vztahuje k~této aplikaci a~k~té dostupné z~url \url{http://oks.kiv.zcu.cz/Prevodnik/Prevodnik}.

Ze scénářů i~z~přiloženého zdrojového kódu je patrné, že přístup k~testování obou aplikací je totožný. Testy vypadají stejně jak pro desktopovou aplikaci, tak pro webovou. Jediné rozdíly nastávají ve způsobu spouštění aplikací, zacházení s~nimi a~v~cestě k~použitým řídícím screenshotům. Příčinou je to, že se jedná o~téměř identicky vypadající a~chovající se aplikace.

	\section{Rozdělení testů}
	Scénáře byly rozděleny do tří částí a~každá část poté může obsahovat další skupiny, které sjednocují tematicky si blízké testy. Struktura tedy vypadá přibližně takto:
		{\renewcommand{\labelenumii}{\theenumii}
		\renewcommand{\theenumii}{\theenumi.\arabic{enumii}.}
		\begin{enumerate}
		\item Statické prvky
			\begin{enumerate}
			\item Přítomnost prvků
			\item Editovatelnost polí
			\item Úplnost výběrových seznamů
			\end{enumerate}
		\item Převody
			\begin{enumerate}
			\item Happy Day Scenario
			\item Stejné jednotky
			\item Varianty Vstup OK
			\item Vše na metr
			\item Vše z~metru
			\item Varianty Vstup CHYBA
			\item Všechny vstupy na všechny výstupy
			\item Hraniční hodnoty
			\end{enumerate}
		\item Vymazání
		\end{enumerate}}
		
	\section{Statické prvky}
	Scénáře v~této části pouze zkontrolují, zda testovaná aplikace obsahuje všechny prvky, jako např. tlačítka či vstupní pole. Dále se zkoumá, zda je vstupní pole editovatelné a~výstupní pole nikoli. Poté se zjistí, zda jsou ve výběrových seznamech obsaženy všechny položky.
	
	\section{Převody}
	V~této části jsou zpracované funkční testy konkrétních převodů. Nejprve se provedou testy Happy Day Scenario -- scénář, kdy vše dopadne podle očekávání. Poté se zkontrolují převody mezi stejnými jednotkami, převody s~možnými korektními i~nekorektními vstupy, převody mezi všemi jednotkami a~nakonec převody s~hraničními hodnotami.
	
	\section{Vymazání}
	V~této části se testuje funkčnosti tlačítka Vymazat. Otestuje se případ, kdy se nevyskytla chybová hláška, i~ten, kdy se vyskytla.
	
	\section{Zjištěné chyby}
	Během testování aplikace bylo zjištěno několik chyb. Tyto chyby jsou v~aplikaci zaneseny záměrně.
	
		\subsection{Chybné převody z~(na) decimetry}
		Pokud provádíme převod z~(případně na) decimetry, dostaneme nesprávný výsledek. Chování odpovídá převodu z~(na) palce. Dále u~jednotek decimetry i~palce v~situaci, kdy jsou použity jak na vstupu, tak na výstupu, je hodnota 3 převedena přibližně na 2.9999996.
		
		\subsection{Převod záporné hodnoty}
		Při zadání záporné hodnoty pro převod se zobrazí chybová hláška o~záporném čísle, avšak převod se i~tak provede.
		
		\subsection{Tlačítko Vymaž}
		Tlačítko vymaž nenastaví všem komponentám výchozí hodnoty. Pouze vymaže obsah vstupního pole. Výstupní pole a~výběrové seznamy nadále obsahují hodnoty z~posledního převodu.