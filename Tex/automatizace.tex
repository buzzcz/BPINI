\chapter{Automatizace nástrojem}
Podle \citep{SikuliX} je SikuliX primárně nástroj pro automatizaci činností. Je vhodný pro kohokoli, kdo provádí opakovaně některé monotónní činnosti jako denní práce s~aplikacemi nebo webovými stránkami, hraní her, administrace IT systémů a~sítí apod.

Zde ukážeme možnost automatizace takového procesu na konkrétním příkladu, viz kód \ref{automatizace}. Jedná se o~zapnutí a~případné přihlášení (pokud jsme se během sezení již nepřihlašovali) do aplikace \emph{TestLink} instalované podle postupu v~podkladech k~přednáškám KIV/OKS, viz \citep{Herout}.

Postup je ve své podstatě identický jako při vytváření testovacích případů. Jediným rozdílem je, že nebude nutné používat knihovnu \emph{JUnit}, neboť neprovádíme testování, ale skript vytvoříme jako samostatný program v~Javě. Každý si ve skriptu musí upravit cestu k~\emph{Bitnami TestLink Stack Manager Tool}, login a~heslo.

\begin{lstjava}{caption={Automatizace spuštění a přihlášení do aplikace \emph{TestLink}}, label={automatizace}}
public static void main(String[] args) {
  Logger logger = LogManager.getLogger();

  Settings.MoveMouseDelay = 0;
  Debug.setLogger(logger);
  Debug.setLoggerAll("info");

  Screen s = new Screen();
  try {
    Runtime.getRuntime().exec("cesta/k/TestLink/manager/
      tool");
    App application = new App("TestLink");
    application.focus();
    s.wait("png/goToApp.png", 5);
    s.click("png/goToApp.png");
    if (s.exists("png/startServers.png") != null) s.
      click("png/startServers.png");
    application = new App("Chrome");
    application.focus();
    s.wait("png/access.png", 50);
    s.click("png/access.png");
    if (s.exists("png/login.png") != null) {
      s.find("png/login.png").below(20).click();
      s.paste("login");
      s.click("png/login.png");
      s.find("png/pass.png").below(20).click();
      s.paste("password");
      s.type(Key.ENTER);
    }
  } catch (IOException | FindFailed e) {
    s.capture().save("errors", screenshotName());
    logger.error(e.getMessage());
  }
}
\end{lstjava}