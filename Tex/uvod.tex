\chapter{Úvod}
Na katedře Informatiky a~výpočetní techniky Fakulty aplikovaných věd Západočeské univerzity je vyučován předmět KIV/OKS -- Ověřování kvality software. Zabývá se testováním software od statického testování, logování, jednotkových testů, strukturálního a~funkčního testování až po testování webových aplikací. Snahou je vždy využít možnosti automatizace testů. Studenti se mají seznámit se základy zajišťování kvality a~testováním softwaru jak teoreticky, tak prakticky. Jednou částí, které není dosud věnována pozornost, je \emph{testování grafického uživatelského rozhraní} aplikací. Pro zaplnění této mezery je potřeba nejdříve nalézt vhodný nástroj a~ověřit jeho možnosti.

Pro testování GUI existuje množství nástrojů jak proprietárních tak i~volně dostupných. Mezi nimi je potřeba provést kvalifikovaný výběr, nejlépe metodou multikriteriálního hodnocení. Zvolený nástroj je poté zapotřebí prozkoumat na úrovni základních postupů práce tak, aby bylo možné překonat počáteční bariéru začátku práce s~tímto nástrojem. Jako další krok je nutné prozkoumat API a~připravit ukázky různých typů automatizovaných testů zapsaných v~programovacím jazyce Java. Všechny tyto akce by měly demonstrovat základní možnosti zvoleného nástroje.

Jedním z~vrcholů práce by pak měla být ucelená sada testů, která bude svou funkčností plně korespondovat s~již existujícími testy webového rozhraní připraveného pomocí jiného typu nástroje (Selenium). Cílem je ukázat, že možnosti zvoleného nástroje jsou minimálně stejné, prakticky však širší než možnosti nástroje pro testování pouze webových aplikací.

Obecně platí, že nástroje pro testování GUI umožňují i~přípravu tzv. monkey testů. Pro to v~práci bude věnována pozornost i~těmto typům testů a~ukázkám jejich možností.

Jako poslední, nikoli však nevýznamná možnost použití, je u~těchto nástrojů i~automatizace rutinních činností. To znamená, že nástroj není použit pro testování, ale pro vytváření jakýchsi \uv{skriptů}, které mohou výrazně usnadnit a~zpřesnit rutinní činnosti  prováděné např. administrátory systémů.