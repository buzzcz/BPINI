\chapter{Závěr}
Seznámil jsem se s~některými metodami testování grafického uživatelského rozhraní a~zjistil jsem některé důvody používání těchto metod. Dále jsem prozkoumal, které nástroje je k~testování možné využít.

Vybral jsem tři programy, které jsem stručně popsal. Navrhl jsem multikriteriální hodnocení a~provedl jejich podrobné porovnání. Výsledkem byl výběr jednoho programu, který použiji jako hlavní nástroj v~této práci.

S~aplikací SikuliX, kterou jsem zvolil předchozí činností, jsem se zběžně seznámil a~vytvořil jeden test v~prostředí SikuliX-IDE za použití vlastního jazyka SikuliX. Další čtyři testy jsem zhotovil pomocí Java API, které SikuliX nabízí a~které budu využívat z~důvodu vazby na předmět KIV/OKS.

Dále se hodlám zaměřit na podrobnější zkoumání používání nástroje. Také připravím sadu ukázkových testů a~funkční scénáře.

\section{Statistika}
V~této práci bylo vytvořeno 103 testů (každý pro webovou a~JavaFX aplikaci, dohromady tedy 206 testů). Počet snímků potřebných pro vytvoření testů byl 74 (každý pro webovou a~JavaFX aplikaci jak na Linux, tak na Windows, celkem tedy 296 snímků). Dále byl vytvořen jeden monkey test a~jeden skript pro ukázku automatizace činnosti. Dohromady bylo vytvořeno více než 330 souborů a~napsáno více než 7350 řádek zdrojového kódu.