\chapter{O testování}
Testování aplikací je nedílnou součástí jejich vývoje a~v~dnešní době se tomuto oddílu tvorby aplikací věnuje čím dál více pozornosti. Dá se rozdělit do různých skupin, např. podle toho, kdy se testování provádí, jakým způsobem se provádí, jak se k~testované aplikaci přistupuje, či jaká část aplikace se podrobuje testům.

Jednou z~důležitých součástí je testování grafického uživatelského rozhraní. Zde se testeři soustředí na to, zda daná aplikace vypadá tak, jak to požadují vývojáři a~návrh, a~zda grafické prvky správně fungují. Dále se zaměřuje na to, zda je aplikace přívětivá k~uživateli a~práce s~ní není příliš komplikovaná.

Při testování grafického uživatelského rozhraní se může spousta testů mnohokrát opakovat, a~proto je snaha tyto testy nějak automatizovat. K~tomu se může využít některý z~nástrojů k~tomu určený. Cílem této práce je seznámit se s~některými z~těchto nástrojů, jeden z~nich vybrat a~pomocí něj vytvořit sadu ukázkových testů svou filosofií zapadajících do předmětu KIV/OKS.