\chapter{SikuliX}
	\section{Instalace}
	Po stažení balíčku započne instalace jeho spuštěním\footnote{Je potřeba instalace JRE nebo JDK 6 a~vyšší, v~linuxové distribuci balíky \emph{libopencv-core2.4, libopencv-imgproc2.4, libopencv-highgui2.4, libtesseract3} a~\emph{wmctrl} \citep{SikuliX}}. Je ukázána instalace v~Linuxu, avšak instalace ve Windows je obdobná. V~průběhu máme na výběr různé možnosti, jak chceme nástroj používat, viz obrázek \ref{Instal}. Např. zda chceme používat SikuliX-IDE a~Python nebo Ruby, jestli budeme používat jiné IDE a~Javu a~zda chceme používat OCR funkce. Zaškrtneme všechna políčka kromě \emph{Ruby (JRuby)} a~klikneme na \emph{Setup Now}. Jsme dotázáni, zda chceme balíčky stáhnout, nebo ukončit instalaci. Zvolíme \emph{Yes}. Další dotaz je na verzi Jythonu, kterou chceme použít, s~upozorněním, že může nastat problém se znaky v~kódování UTF-8. Opět zvolíme \emph{Yes}. Začne vytváření souborů a~měla by se otevřít dvě okna jako na obrázu \ref{InstalOK}, obě potvrdíme tlačítkem \emph{OK}. Pokud vše proběhne v~pořádku, vzniknou v~adresáři soubory podobné těmto\footnote{Může se lišit na různých OS} \emph{runsikulix, SetupStuff, SikuliX-1.1.0-SetupLog.txt, sikulixapi.jar, sikulix.jar}.
	\begin{figure}[ht!]
		\centering
		\includegraphics[width=13.5cm]{img/Instalace/Instalace.png}
		\caption{Instalace SikuliX}
		\label{Instal}
	\end{figure}
	\begin{figure}[ht!]
		\centering
		\includegraphics[width=9cm]{img/Instalace/InstalaceOK.png}\\[0.3cm]
		\includegraphics[width=9cm]{img/Instalace/InstalaceOK1.png}
		\caption{Test instalace}
		\label{InstalOK}
	\end{figure}
	
	\section{Tvorba testů}
	Pro tvorbu testů pomocí SikuliX jsou nejdůležitější snímky (screenshoty) řídících prvků, které bude SikuliX hledat a~případně používat k~některým akcím. Je tedy vhodné si nejprve aplikaci spustit, vybrat příslušné prvky a~vytvořit jejich snímky. Při jejich tvorbě se doporučuje preciznost a~přesnost, neboť v~jistých situacích mohou nastat problémy, které budou zmíněny později.
	
	Pokud je vytvářený test jednoduchý a~není potřeba většího množství testů, je jednodušší vytvořit jej pomocí SikuliX-IDE. Pokud však chceme aplikaci testovat podrobněji a~psát velké množství testovacích případů, je vhodnější použít některý z~podporovaných programovacích jazyků a~využít tak jeho možnosti jako nadstavbu nad SikuliX.
	
	\section{SikuliX-IDE}
	Spustit SikuliX-IDE je možné různými způsoby \citep{SikuliX}.
	\begin{enumerate}
		\item Spuštěním souboru SikuliX.app (Mac) nebo SikuliX.exe (Windows),
		\item dvojklikem na soubor runsikulix (Linux) nebo runsikulix.cmd (Windows),
		\item z~příkazové řádky příkazem\\
		\texttt{java -jar cesta/k/sikulix.jar [volitelne parametry]}
	\end{enumerate}
	Po spuštění vypadá IDE jako na obrázku \ref{SikuliXIDE}. Jako parametry se v~metodách, ve kterých je to možné, ukazují obrázky vzorů, podle kterých se na obrazovce nástroj orientuje.
	\begin{figure}[ht!]
		\centering
		\includegraphics[width=13.5cm]{img/IDE/SikuliXIDE.png}
		\caption{SikuliX-IDE}
		\label{SikuliXIDE}
	\end{figure}
	
		\subsection{První skript}
		Skript se připravuje v~SikuliX-IDE, které je vidět na obrázku \ref{SikuliXIDE}. Kód, který je vidět v~\ref{PrvniSkript}, není v~IDE identický, ale cesta k~obrázku je vždy nahrazena jeho náhledem. K~tvorbě jsou v~IDE užitečné pomůcky, které se nacházejí v~levém a~v~horním panelu.
		
		Skript pracuje tak, že se otevře prohlížeč, který přejde na adresu \url{http://oks.kiv.zcu.cz/Prevodnik}. Klikne na odkaz \emph{Převodník}, do vstupního pole vloží \emph{1} a~stiskne \emph{Převeď}. Z~pole s~výsledkem přečte text a~porovná jej s~předpokládanou hodnotou \emph{2,54}. Pokud si odpovídají, objeví se dialogové okno s~potvrzením, jestliže ne, zobrazí se chybová hláška. Obdobně je tomu v~následující části, kde se pouze kontroluje existence obrázku.
		
	\section{Java API}
	Dále bylo zkoumáno Java API, které SikuliX poskytuje. Pro jeho použití je potřeba mít při překladu a~spuštění nastavený v~classpath \emph{sikulixapi.jar}. Toho docílíme např. tak, že použijeme v~příkazové řádce dvou příkazů\\\texttt{javac -cp sikulixapi.jar:. Test01.java}\\\texttt{java -cp sikulixapi.jar:. Test01}\\ Syntaxe, kterou SikuliX v~Java API využívá, je velmi podobná té v~SikuliX-IDE.
	
		\subsection{První test}
		První test s~použitím Java API, viz kód \ref{PrvniJavaAPI}, je téměř identický s~tím, který byl vytvořen pomocí SikuliX-IDE.
	
		\subsection{Sofistikovanější testy}
		S~využitím knihoven \emph{JUnit} a~\emph{Log4j} (ani jedna z~těchto knihoven není pro běh SikuliX bezprostředně nutná) byly vytvořeny čtyři testy, viz kód \ref{DalsiJavaAPI}. První test skončí negativně, druhý pozitivně, třetí pozitivně a~čtvrtý negativně.