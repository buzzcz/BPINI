\begin{lstpython}{caption={První skript}, label={PrvniSkript}}
#prohlizec je promenna
prohlizec = App("google-chrome")
prohlizec.open()	#otevre aplikaci definovanou
				#vyse
prohlizec.focus()	#vybere do popredi jeji okno
#ceka, dokud na obrazovce nenajde obrazek
wait("obr1.png")
#najde na obrazovce obrazek a vlozi do neho text
paste("obr2.png", "http://oks.kiv.zcu.cz/Prevodnik")
type(Key.ENTER)	#simuluje stisk klavesy ENTER
#najde na obrazovce obrazek a klikne na nej
click("obr3.png")
wait("obr4.png")
paste("obr5.png", "1")
#klikne o 27px vyse a 18px vlevo od nalezeneho
#obrazku
click(Pattern("obr6.png").targetOffset(-27,-18))
click("obr7.png")
click("obr4.png")
#precte text z casti, ktera je vpravo od nalezeneho
#obrazku 100px siroka, do promenne vystup
vystup = find("obr8.png").right(100).text()
if vystup == "2.54":
	#pokud rozpoznany text souhlasi se zadanym,
	#otevre se vyskakovaci okno
    popup("Ok textove")
else:
    popError("Chyba")    #jinak se zobrazi chybove
    			 #okno

if exists("obr9.png"):
	#pokud na obrazovce existuje obrazek, otevre
	#se vyskakovaci okno
    popup("Ok obrazove")
else:
    popError("Chyba")
prohlizec.close()    #ukonci aplikaci
\end{lstpython}