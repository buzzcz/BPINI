%%%%%%%%%%%%%%%%%%%%%%%%%%%%%%%%%%%%%%%%%%%%%%%%%%%%%%%%%%
%
% Vzor pro sazbu kvalifikační práce
%
% Západočeská univerzita v Plzni
% Fakulta aplikovaných věd
% Katedra informatiky a výpočetní techniky
%
% Petr Lobaz, lobaz@kiv.zcu.cz, 2016/03/14
%
%%%%%%%%%%%%%%%%%%%%%%%%%%%%%%%%%%%%%%%%%%%%%%%%%%%%%%%%%%

% Možné jazyky práce: czech, english
% Možné typy práce: BP (bakalářská), DP (diplomová)
\documentclass[czech,BP]{thesiskiv}

% Definujte údaje pro vstupní strany
%
% Jméno a příjmení; kvůli textu prohlášení určete, 
% zda jde o mužské, nebo ženské jméno.
\author{Jaroslav Klaus}
\declarationmale

%alternativa: 
%\declarationfemale

% Název práce
\title{Využití nástrojů pro testování grafického uživatelského rozhraní}

% 
% Texty abstraktů (anglicky, česky)
%
\abstracttexten{This paper deals with the use of software tools for testing graphical user interface. It compares some of the tools and describes the use of one of them in a~way that fits into the subject KIV/OKS.}

\abstracttextcz{Tato práce se zabývá využitím nástrojů k~testování grafického uživatelského rozhraní aplikací. Srovnává některé nástroje k~tomu určené a~popisuje použití jednoho z~nich tak, aby svou filosofií zapadal do předmětu KIV/OKS.}

% Na titulní stranu a do textu prohlášení se automaticky vkládá 
% aktuální rok, resp. datum. Můžete je změnit:
%\titlepageyear{2016}
%\declarationdate{1. března 2016}

% Ve zvláštních případech je možné ovlivnit i ostatní texty:
%
%\university{Západočeská univerzita v Plzni}
%\faculty{Fakulta aplikovaných věd}
%\department{Katedra informatiky a výpočetní techniky}
%\subject{Bakalářská práce}
%\titlepagetown{Plzeň}
%\declarationtown{Plzni}

%%%%%%%%%%%%%%%%%%%%%%%%%%%%%%%%%%%%%%%%%%%%%%%%%%%%%%%%%%
%
% DODATEČNÉ BALÍČKY PRO SAZBU
% Jejich užívání či neužívání záleží na libovůli autora 
% práce
%
%%%%%%%%%%%%%%%%%%%%%%%%%%%%%%%%%%%%%%%%%%%%%%%%%%%%%%%%%%
\usepackage{float}
\usepackage{eurosym}
\usepackage{longtable}
\usepackage{subcaption}
\usepackage{enumitem}
\usepackage{listings}
\usepackage{color}

\definecolor{brown}{rgb}{.33,.28,0}
\definecolor{gray}{rgb}{.3,.42,.45}
\definecolor{darkRed}{rgb}{.51,0,.12}
\definecolor{tyrk}{rgb}{.4,.8,.75}
\definecolor{darkGreen}{rgb}{0,.51,.12}

\renewcommand{\lstlistingname}{Kód}

\lstnewenvironment{lstpython}[2][]{\lstset{
	#1,
	#2,
	captionpos=b,
	language=Python,
	frame=single,
	basicstyle=\ttfamily,
	showstringspaces=false,
	stringstyle=\color{darkRed},
	commentstyle=\color{gray},
	numberstyle=\color{brown},
	keywordstyle=\color{tyrk},
	morekeywords={open, focus, wait, paste, type, click, find, popup, popError, exists, close, right, text, targetOffset},
	keywordstyle=[2]\color{red},
	keywords=[2]{App, Key},
	keywordstyle=[3]\color{brown},
	keywords=[3]{ENTER}
}}{}
\lstnewenvironment{lstjava}[2][]{\lstset{
	#1,
	#2,
	captionpos=b,
	language=Java,
	frame=single,
	basicstyle=\ttfamily,
	showstringspaces=false,
	commentstyle=\color{darkGreen},
	keywordstyle=\color{blue},
	stringstyle=\color{red}
}}{}

% Zařadit literaturu do obsahu
\usepackage[nottoc,notlot,notlof]{tocbibind}

% Umožňuje vkládání obrázků
\usepackage[pdftex]{graphicx}

% Odkazy v PDF jsou aktivní; navíc se automaticky vkládá
% balíček 'url', který umožňuje např. dělení slov
% uvnitř URL
\usepackage[pdftex]{hyperref}
\hypersetup{colorlinks=true,
  unicode=true,
  linkcolor=black,
  citecolor=black,
  urlcolor=black,
  bookmarksopen=true}

% Při používání citačního stylu csplainnatkiv
% (odvozen z csplainnat, http://repo.or.cz/w/csplainnat.git)
% lze snadno modifikovat vzhled citací v textu
\usepackage[square,sort&compress]{natbib}

%%%%%%%%%%%%%%%%%%%%%%%%%%%%%%%%%%%%%%%%%%%%%%%%%%%%%%%%%%
%
% VLASTNÍ TEXT PRÁCE
%
%%%%%%%%%%%%%%%%%%%%%%%%%%%%%%%%%%%%%%%%%%%%%%%%%%%%%%%%%%
\begin{document}
%
\maketitle
\cleardoublepage
\pagenumbering{gobble}
\tableofcontents
\cleardoublepage
\pagenumbering{arabic}

\chapter{Úvod}
Na katedře Informatiky a~výpočetní techniky Fakulty aplikovaných věd Západočeské univerzity je vyučován předmět KIV/OKS -- Ověřování kvality software. Zabývá se testováním software od statického testování, logování, jednotkových testů, strukturálního a~funkčního testování až po testování webových aplikací. Snahou je vždy využít možnosti automatizace testů. Studenti se mají seznámit se základy zajišťování kvality a~testováním softwaru jak teoreticky, tak prakticky. Jednou částí, které není dosud věnována pozornost, je \emph{testování grafického uživatelského rozhraní} aplikací. Pro zaplnění této mezery je potřeba nejdříve nalézt vhodný nástroj a~ověřit jeho možnosti.

Pro testování GUI existuje množství nástrojů jak proprietárních tak i~volně dostupných. Mezi nimi je potřeba provést kvalifikovaný výběr, nejlépe metodou multikriteriálního hodnocení. Zvolený nástroj je poté zapotřebí prozkoumat na úrovni základních postupů práce tak, aby bylo možné překonat počáteční bariéru začátku práce s~tímto nástrojem. Jako další krok je nutné prozkoumat API a~připravit ukázky různých typů automatizovaných testů zapsaných v~programovacím jazyce Java. Všechny tyto akce by měly demonstrovat základní možnosti zvoleného nástroje.

Jedním z~vrcholů práce by pak měla být ucelená sada testů, která bude svou funkčností plně korespondovat s~již existujícími testy webového rozhraní připraveného pomocí jiného typu nástroje (Selenium). Cílem je ukázat, že možnosti zvoleného nástroje jsou minimálně stejné, prakticky však širší než možnosti nástroje pro testování pouze webových aplikací.

Obecně platí, že nástroje pro testování GUI umožňují i~přípravu tzv. monkey testů. Pro to v~práci bude věnována pozornost i~těmto typům testů a~ukázkám jejich možností.

Jako poslední, nikoli však nevýznamná možnost použití, je u~těchto nástrojů i~automatizace rutinních činností. To znamená, že nástroj není použit pro testování, ale pro vytváření jakýchsi \uv{skriptů}, které mohou výrazně usnadnit a~zpřesnit rutinní činnosti  prováděné např. administrátory systémů.

\chapter{Testování softwaru}
%SMAZAT!!
%Testování aplikací je nedílnou součástí jejich vývoje a~v~dnešní době se tomuto oddílu tvorby aplikací věnuje čím dál více pozornosti. Dá se rozdělit do různých skupin, např. podle toho, kdy se testování provádí, jakým způsobem se provádí, jak se k~testované aplikaci přistupuje, či jaká část aplikace se podrobuje testům.

%Jednou z~důležitých součástí je testování grafického uživatelského rozhraní. Zde se testeři soustředí na to, zda daná aplikace vypadá tak, jak to požadují vývojáři a~návrh, a~zda grafické prvky správně fungují. Dále se zaměřuje na to, zda je aplikace přívětivá k~uživateli a~práce s~ní není příliš komplikovaná.

%Při testování grafického uživatelského rozhraní se může spousta testů mnohokrát opakovat, a~proto je snaha tyto testy nějak automatizovat. K~tomu se může využít některý z~nástrojů k~tomu určený. Cílem této práce je seznámit se s~některými z~těchto nástrojů, jeden z~nich vybrat a~pomocí něj vytvořit sadu ukázkových testů svou filosofií zapadajících do předmětu KIV/OKS.

Na začátek je potřeba vysvětlit některé pojmy z~oblasti testování. Budeme vycházet hlavně z~\citep{RizeniKvalitySW}.

	\section{Požadavky}
	Požadavky zachycují přání zákazníka na funkcionalitu softwaru. Dělí se na dvě skupiny:
		\begin{itemize}
			\item Funkční -- popisují funkčnost služby vykonávané systémem, tedy co má vykonávat. Patří sem např.:
				\begin{itemize}
					\item Uživatel bude moci vytvořit záznam pro nového zákazníka.
					\item Systém automaticky odhlásí uživatele po 3 minutách nečinnosti.
				\end{itemize}
			\item Mimofunkční -- popisují určité vlastnosti systému, či omezující podmínky. V podstatě říkají, jaký by systém měl být. Sem patří např.:
				\begin{itemize}
					\item Modul "Správa klientů" bude dostupný pouze uživatelům s~rolí správce.
					\item Systém bude použitelný při zátěži 1000 uživatelů.
				\end{itemize}
		\end{itemize}
	
	Je vhodné, aby se testeři zabývali i~požadavky, neboť mohou již v~rané fázi vývoje zachytit ty chybně formulované (nekonzistentní, neproveditelné, nekompletní, netestovatelné, nejednoznačné, více požadavků zapsaných jako jeden apod.).
	
	\section{Specifikace požadavků na software}
	Funkční i~mimofunkční požadavky zákazníka, jejich analýza a~dokumentace a~všeobecný popis systému se zapisuje do dokumentu nazvaného specifikace požadavků na software. Na základě tohoto dokumentu probíhají následující fáze vývoje, proto je jeho správnost velmi podstatná.
	
	K~tomuto dokumentu se poté vztahuje i~testování, konkrétně funkční testování, které kontroluje, zda software vyhovuje a~splňuje požadavky zákazníka.
	
	\section{Kvalita softwaru}
	Kvalita softwaru je velmi obtížně definovatelný pojem. Pro její definici vzniklo několik norem. Ty jsou dnes zastaralé či nekonzistentní, proto jsou nahrazovány jednotným systémem norem ISO/IEC 25000-25099 v~rámci projektu SQuaRe (Software Quality Requirements and Evaluation).
	
	Např. norma ISO/IEC 25010 říká, že kvalita softwaru je míra, do jaké softwarový produkt splňuje stanovené a~implicitní potřeby, je-li používán za stanovených podmínek.
	
		\subsection{FURPS}
		Dnes nejčastějším modelem kvality softwaru je tzv. FURPS, který vytvořila společnost Hewlett-Packard. Ten kvalitu popisuje pomocí těchto pěti charakteristik:
			\begin{itemize}
				\item Funkčnost -- soubor požadované funkcionality, schopností a~bezpečnostních aspektů systému.
				\item Použitelnost -- snadnost použití, konzistence, estetika, dokumentace apod.
				\item Spolehlivost -- četnost a~závažnost selhání, doba bezporuchového běhu, správnost výstupů, zotavení atd.
				\item Výkonnost -- odezva systému, výkon za různých podmínek, požadavky na systémové prostředí.
				\item Rozšiřitelnost/podporovatelnost -- škálovatelnost, udržovatelnost, testovatelnost, snadnost konfigurace.
			\end{itemize}
		Většinou se setkáme s~modelem FURPS+, který navíc přidává kategorie jako omezení návrhu, požadavky na implementaci, požadavky na rozhraní a~požadavky na fyzické vlastnosti.
			
	\section{Chyba, defekt, selhání}
	Během vývoje softwaru se do dokumentů či zdrojových kódů dostávají defekty, způsobující chyby a~selhání. Tyto pojmy je velmi důležité rozlišovat. V~následujících odstavcích bude jejich význam vysvětlen.
	
	Selhání nastává v~případě, že jeden nebo více výstupních stavů aplikace se odlišuje od stavu správného (nesplňuje specifikaci, případně specifikace nebyla kompletní nebo jednoznačná).
	
	Právě toto odchýlení od očekávaného stavu se nazývá chybou. Původ chyby se nazývá defekt a~označuje se tak vada v~kódu či datech. Nejčastěji jej způsobí programátor chybou v~kódu, špatným návrhem, nedostatečně či nesprávně pochopenou specifikací, nebo záměrnou sabotáží.
	
	Jednotlivé pojmy na sebe tedy navazují následovně (viz obrázek \ref{Bug}): defekt (aktivace) $\to$ chyba (šíření) $\to$ selhání (příčina) $\to$ defekt\dots
	\begin{figure}[ht!]
		\centering
		\includegraphics[width=8cm]{img/Bug.png}
		\caption{Šíření chyby mezi systémy. Selhání systému A je pro příjemce jeho služby (systém B) externím defektem, který může vést k~chybě a~ta poté opět  k~selhání. Zdroj \citep{RizeniKvalitySW}}
		\label{Bug}
	\end{figure}

\chapter{Přehled nástrojů}
V~této kapitole následuje přehled nástrojů a~některých jejich vlastností. V~tabulce \ref{PrehledNastroju} je uveden název nástroje, jeho licence resp. cena, jazyk, ve kterém se testy píší, platforma, na které nástroj funguje a~která GUI je nástroj schopen testovat. Z~bezplatných multiplatformních nástrojů jsem si vybral tři a~ty podrobněji prozkoumal a~porovnal, viz následující kapitola.
{\scriptsize
\begin{longtable}{|l|l|l|l|l|l|}
		\hline
		\textbf{Název}&\textbf{Licence/Cena}&\textbf{Skriptovací jazyk}&\textbf{Platforma}&\textbf{\shortstack{\\Jazykové\\omezení}}\\\hline\hline
		AutoIt \cite{AutoIt}&Freeware&BASIC-like&Windows&-\\\hline
		AutoHotKey \citep{AutoHotKey}&GNU GPLv2&AutoHotKey&Windows&-\\\hline
		AutoKey \citep{AutoKey}&GNU GPLv3&Python&Linux&-\\\hline
		\shortstack{\\SikuliX \citep{Sikuli}\\\citep{SikuliX}}&MIT License&Python, Ruby&\shortstack{\\Windows,\\Linux, Mac}&-\\\hline
		Jubula \citep{Jubula}&EPL 1.0&\shortstack{\\Drag \& Drop,\\Java}&\shortstack{\\Windows,\\Linux, Mac}&\shortstack{\\Java, HTML,\\.NET, iOS} \\\hline
		\shortstack{Robot\\Framework\\\citep{RobotFramework}}&\shortstack{Apache\\License 2.0}&Natural-like&\shortstack{\\Windows,\\Linux, Mac}&\shortstack{\\Podle pluginů\\(Java, web,\\Android, iOS,\\\dots)}\\\hline
		Squish \citep{Squish}&\shortstack{\\cca\\\EUR{2400}/osoba}&\shortstack{\\Python, JavaScript,\\Ruby, Perl, Tcl}&\shortstack{\\Windows,\\Linux, Mac}&-\\\hline
		eggPlant \citep{eggPlant}&\shortstack{\\nedostupná,\\vázaná na stroj}&\shortstack{\\SmartTalk,\\Drag \& Drop,\\pomocí rozhraní\\eggDrive např.\\Java, C\#, Ruby}&\shortstack{\\Windows,\\Linux, Mac}&-\\\hline
		UFT \citep{UFT}&nedostupná&\shortstack{\\VBScript,\\Drag \& Drop}&\shortstack{\\Windows,\\Linux, Mac}&-\\\hline
		\shortstack{\\Rational\\Functional\\Tester} \citep{RFT}&3300 \$/osoba&Nahrávání akcí&\shortstack{\\Windows,\\Linux}&-\\\hline
		Ranorex \citep{Ranorex}&\EUR{690}&\shortstack{\\C\#, VisualBasic,\\nahrávání akcí}&Windows&-\\\hline
		SilkTest \citep{SilkTest}&nedostupná&\shortstack{\\C\#, VisualBasic,\\Java}&Windows&-\\\hline
		\shortstack{\\TestComplete\\\citep{TestComplete}}&\EUR{889 }/stroj&\shortstack{\\Python, VBScript,\\JScript, C\#Script,\\DelphiScript,\\C++Script,\\nahrávání akcí}&Windows&-\\\hline
	\captionsetup{font=normalsize}
	\caption{Přehled nástrojů}
	\label{PrehledNastroju}
\end{longtable}
}

\chapter{Zvolené nástroje}
Vzhledem k~požadavkům na nástroje, které vyplývají z~vazby na předmět KIV/OKS, jako je bezplatnost, schopnost fungování nezávisle na OS nebo podpora testování programů vytvořených technologií Java a~webových aplikací, jsem z~výše zmíněných vybral nástroje Jubula, SikuliX a~Robot Framework. Každý z~nástrojů bude stručně charakterizován a~bude následovat podrobnější srovnání.
	\section{Jubula}
	Jubula je nástroj, který vznikl a~je vyvíjen v~rámci IDE Eclipse. Do projektu přispívá také firma BREDEX GmbH, která vytváří i~tzv. standalone verzi, což je program, který je možné používat samostatně bez IDE Eclipse. Navíc obsahuje navíc některé nespecifikované funkce a~nemusí být licencována pod EPL 1.0, jako je tomu u~verze pro IDE Eclipse.
	
	Pro tvorbu testovacích skriptů byla používána metoda Drag \& Drop, popř. se akce určovaly klikáním na různé nabídky. V~jedné z~posledních verzí bylo vydáno Java API a~skripty je tak možné psát pomocí jazyka Java. Mezi podporovaná testovaná rozhraní patří Java Swing, SWT, JavaFX, HTML a~iOS. Výhodou této aplikace je také možnost její integrace do ostatních programů pro organizaci testování.
	
	\section{SikuliX}
	Sikuli (nověji SikuliX) je nástroj, který vznikl jako projekt skupiny User Interface Design Group na MIT, což odpovídá i~jeho licenci -- MIT License. Nyní jeho vývoj převzal Raimund Hock (aka RaiMan) společně s~open-source komunitou.
	
	Při tvorbě skriptů je možné využít pro SikuliX vlastní jazyk podobný přirozené angličtině, nebo některý ze zavedených, jako je Python, Ruby, Java, Jython, JRuby, Scala, Groovy, Clojure a~další. Nástroj není omezený na určitá testovaná rozhraní, protože k~identifikaci GUI používá rozpoznávání obrazu podle vzoru\footnote{Pomocí OpenCV, \url{http://opencv.org/}}, dokáže simulovat ovládání myši a~klávesnice nebo rozpoznávat text v~obrázcích\footnote{Pomocí Tesseract OCR, \url{https://github.com/tesseract-ocr}}. Výhodou této aplikace je proto její nezávislost vůči testovanému rozhraní. Cenou za to je pravděpodobné snížení její rychlosti. Použití SikuliX se neomezuje pouze na testování, ale je možné pomocí něj i~automatizovat činnosti.
	
	\section{Robot Framework}
	Robot Framework je nástroj založený na pluginech a~je open-source. Vývoj podporuje společnost Nokia Networks.
	
	Základ nástroje, tzv. core framework, je vytvořený v~jazyce Python. Knihovny je možné psát v~jazyce Python nebo Java a~samotné skripty pak v~jazyce podobném přirozené angličtině. Díky dodržování jistého formátování je pro člověka velmi přehledný. Mezi podporovaná testovaná rozhraní patří např. Android, iOS, Java Swing, webové aplikace, databáze a~aplikace vytvořené pro OS Windows. Výhodou této aplikace je možnost si chybějící modul pro testování určitého rozhraní vytvořit a~používat.

\chapter{Srovnání nástrojů}
Pro srovnání nástrojů jsem vytvořil návrh multikriteriálního hodnocení, který se snaží nástroje hodnotit z~různých úhlů a~vytvořit tak komplexní klasifikaci. Každé z~hodnocených částí je možné přiřadit vlastní váhu. Ta určuje důležitost hodnotícího kritéria pro každého jedince a~tím napomáhá výběru vhodného nástroje. V~obrázku \ref{MKHodn} je návrh ukázán a~je vidět výsledek pro mnou zvolené váhy, viz obrázek \ref{MKVysl}. Jako nejvhodnější se jeví použití nástroje SikuliX. Dále se budu věnovat jednotlivým hodnotícím kritériím.

Možnost vytváření skriptů je jedno z~nejdůležitějších kritérií vzhledem k~vazbě na předmět KIV/OKS. Hlavním požadavkem bylo, aby bylo možné skripty tvořit v~jazyce Java. Dále jsem vybral několik skriptovacích jazyků a~metod.

Podpora testovaných rozhraní byla dalším z~rozhodujících kritérií. Hlavními platformami měly být aplikace vytvořené pomocí jazyka Java a~webové aplikace. Opět jsem přidal některé další běžné platformy. Nástroj by měl být též multiplatformní, proto je jedním z~kritérií podpora operačních systémů.

Reportování výsledků testů, složitost jejich tvorby a~jejich přehlednost může napomoci vývojáři diagnostikovat případnou chybu. Také je přínosné znát stav obrazovky a~to zajistí screenshot. Díky tomu se stává vývoj jednodušší, a~proto jsem toto kritérium také zařadil do hodnocení.

Dále jsem přidal kritérium univerzálnosti nástroje s~poněkud individuálním ohodnocením. To je zde myšleno tak, co obecně nástroj dokáže, ale co není podstatné z~pohledu předmětu KIV/OKS. Například Jubula je čistě testovací nástroj, ale SikuliX se dá použít navíc pro automatizaci pracovních postupů.

Posledním kritériem je vhodnost nástroje pro účely předmětu KIV/OKS. Jedná se hlavně o~to, jak zapadá do konceptu výuky, jak je práce s~ním složitá a~jaké má nároky na studentovy znalosti.

Vysvětlivky termínů z~dále použitých tabulek:
\begin{itemize}
	\item Natural-like -- víceméně okleštěný přirozený jazyk založený na angličtině,
	\item Složitost tvorby -- míní se tím složitost přípravy reportu a~v~tabulce vyšší počet bodů znamená jednodušší tvorbu pro tvůrce skriptů
\end{itemize}
\begin{figure}[ht!]
	\begin{subfigure}{\textwidth}
		\centering
		\includegraphics[width=13.5cm]{img/Kriteria/Kriteria1.png}
		\includegraphics[width=13.5cm]{img/Kriteria/Kriteria2.png}
		\caption{Multikriteriální hodnocení}
		\label{MKHodn}
	\end{subfigure}
\end{figure}
\begin{figure}[ht!]
	\ContinuedFloat
	\begin{subfigure}{\textwidth}
		\centering
		\includegraphics[width=13.5cm]{img/Kriteria/Kriteria3.png}
		\caption{Výsledek multikriteriálního hodnocení}
		\label{MKVysl}
	\end{subfigure}
\end{figure}

Z~uvedeného multikriteriálního hodnocení je zřejmé, že vybrané nástroje jsou v~podstatě vyrovnané. To ostatně potvrzuje i~jejich poměrné zastoupení mezi uživateli. SikuliX byl zvolen též po diskuzích s~vedoucím práce a~to pro jeho vlastnost naprosté nezávislosti na testovaném rozhraní. Tato vlastnost se významně hodí v~předmětu KIV/OKS, protože je možné dát nástroj do kontrastu s~nástrojem Selenium. Jinými slovy řečeno SikuliX je principiálně odlišný nástroj, což není možné říci o~např. Jubule.

\chapter{SikuliX}
	\section{Instalace}
	Po stažení balíčku započne instalace jeho spuštěním\footnote{Je potřeba instalace JRE nebo JDK 6 a~vyšší, v~linuxové distribuci balíky \emph{libopencv-core2.4, libopencv-imgproc2.4, libopencv-highgui2.4, libtesseract3} a~\emph{wmctrl} \citep{SikuliX}}. Je ukázána instalace v~Linuxu, avšak instalace ve Windows je obdobná. V~průběhu máme na výběr různé možnosti, jak chceme nástroj používat, viz obrázek \ref{Instal}. Např. zda chceme používat SikuliX-IDE a~Python nebo Ruby, jestli budeme používat jiné IDE a~Javu a~zda chceme používat OCR funkce. Zaškrtneme všechna políčka kromě \emph{Ruby (JRuby)} a~klikneme na \emph{Setup Now}. Jsme dotázáni, zda chceme balíčky stáhnout, nebo ukončit instalaci. Zvolíme \emph{Yes}. Další dotaz je na verzi Jythonu, kterou chceme použít, s~upozorněním, že může nastat problém se znaky v~kódování UTF-8. Opět zvolíme \emph{Yes}. Začne vytváření souborů a~měla by se otevřít dvě okna jako na obrázu \ref{InstalOK}, obě potvrdíme tlačítkem \emph{OK}. Pokud vše proběhne v~pořádku, vzniknou v~adresáři soubory podobné těmto\footnote{Může se lišit na různých OS} \emph{runsikulix, SetupStuff, SikuliX-1.1.0-SetupLog.txt, sikulixapi.jar, sikulix.jar}.
	\begin{figure}[ht!]
		\centering
		\caption{Instalace SikuliX}
		\label{Instal}
		\includegraphics[width=13.5cm]{img/Instalace/Instalace.png}
	\end{figure}
	\begin{figure}[ht!]
		\centering
		\caption{Test instalace}
		\label{InstalOK}
		\includegraphics[width=9cm]{img/Instalace/InstalaceOK.png}\\[0.3cm]
		\includegraphics[width=9cm]{img/Instalace/InstalaceOK1.png}
	\end{figure}
	
	\section{Tvorba testů}
	Pro tvorbu testů pomocí SikuliX jsou nejdůležitější snímky (screenshoty) řídících prvků, které bude SikuliX hledat a~případně používat k~některým akcím. Je tedy vhodné si nejprve aplikaci spustit, vybrat příslušné prvky a~vytvořit jejich snímky. Při jejich tvorbě se doporučuje preciznost a~přesnost, neboť v~jistých situacích mohou nastat problémy, které budou zmíněny později.
	
	Pokud je snímků více, je vhodné je třídit do adresářů. To není nutné, ale zlepšuje to čitelnost kódu a~usnadňuje práci s~nástrojem. Adresáře mohou např. sdružovat snímky prvků, které jsou si nějakým způsobem podobné (tlačítka, textová pole, výběrové seznamy, chybové hlášky, apod.). Stejně tak je vhodné snímky pojmenovávat na základě toho, co obsahují (vstupní textové pole, label Vstup, apod.).
	
	Dle \citep{SikuliXImgs} SikuliX interně používá třídu ImageIO z~Javy. Podporované formáty jsou tedy bmp, wbmp, jpg, jpeg, png a gif.
	
	Pokud je vytvářený test jednoduchý a~není potřeba většího množství testů, je jednodušší vytvořit jej pomocí SikuliX-IDE. Pokud však chceme aplikaci testovat podrobněji a~psát velké množství testovacích případů, je vhodnější použít některý z~podporovaných programovacích jazyků a~využít tak jeho možnosti jako nadstavbu nad SikuliX.
	
	\section{SikuliX-IDE}
	Spustit SikuliX-IDE je možné různými způsoby \citep{SikuliX}.
	\begin{enumerate}
		\item Spuštěním souboru SikuliX.app (Mac) nebo SikuliX.exe (Windows),
		\item dvojklikem na soubor runsikulix (Linux) nebo runsikulix.cmd (Windows),
		\item z~příkazové řádky příkazem\\
		\texttt{java -jar cesta/k/sikulix.jar [volitelne parametry]}
	\end{enumerate}
	Po spuštění vypadá IDE jako na obrázku \ref{SikuliXIDE}. Jako parametry se v~metodách, ve kterých je to možné, ukazují obrázky vzorů, podle kterých se na obrazovce nástroj orientuje, případně cesta k~nim.
	\begin{figure}[ht!]
		\centering
		\caption{SikuliX-IDE}
		\label{SikuliXIDE}
		\includegraphics[width=13.5cm]{img/SikuliXIDE.png}
	\end{figure}
	
		\subsection{První skript}
		Skript se připravuje v~SikuliX-IDE, které je vidět na obrázku \ref{SikuliXIDE}. Kód, který je vidět v~\ref{PrvniSkript}, není v~IDE identický, ale cesta k~obrázku je vždy nahrazena jeho náhledem. K~tvorbě jsou v~IDE užitečné pomůcky, které se nacházejí v~levém a~v~horním panelu.
		
		Skript pracuje tak, že se otevře prohlížeč, který přejde na adresu \url{http://oks.kiv.zcu.cz/Prevodnik}. Klikne na odkaz \emph{Převodník}, do vstupního pole vloží \emph{1} a~stiskne \emph{Převeď}. Z~pole s~výsledkem přečte text a~porovná jej s~předpokládanou hodnotou \emph{2,54}. Pokud si odpovídají, objeví se dialogové okno s~potvrzením, jestliže ne, zobrazí se chybová hláška. Obdobně je tomu v~následující části, kde se pouze kontroluje existence obrázku.
		
			\begin{lstpython}{caption={První skript}, label={PrvniSkript}}
#prohlizec je promenna
prohlizec = App("google-chrome")
prohlizec.open()	#otevre aplikaci definovanou
				#vyse
prohlizec.focus()	#vybere do popredi jeji okno
#ceka, dokud na obrazovce nenajde obrazek
wait("obr1.png")
#najde na obrazovce obrazek a vlozi do neho text
paste("obr2.png", "http://oks.kiv.zcu.cz/Prevodnik")
type(Key.ENTER)	#simuluje stisk klavesy ENTER
#najde na obrazovce obrazek a klikne na nej
click("obr3.png")
wait("obr4.png")
paste("obr5.png", "1")
#klikne o 27px vyse a 18px vlevo od nalezeneho
#obrazku
click(Pattern("obr6.png").targetOffset(-27,-18))
click("obr7.png")
click("obr4.png")
#precte text z casti, ktera je vpravo od nalezeneho
#obrazku 100px siroka, do promenne vystup
vystup = find("obr8.png").right(100).text()
if vystup == "2.54":
	#pokud rozpoznany text souhlasi se zadanym,
	#otevre se vyskakovaci okno
    popup("Ok textove")
else:
    popError("Chyba")    #jinak se zobrazi chybove
    			 #okno

if exists("obr9.png"):
	#pokud na obrazovce existuje obrazek, otevre
	#se vyskakovaci okno
    popup("Ok obrazove")
else:
    popError("Chyba")
prohlizec.close()    #ukonci aplikaci
\end{lstpython}
		
	\section{Java API}
	Dále bylo zkoumáno Java API, které SikuliX poskytuje. Pro jeho použití je potřeba mít při překladu a~spuštění nastavený v~classpath \emph{sikulixapi.jar}. Toho docílíme např. tak, že použijeme v~příkazové řádce dvou příkazů\\\texttt{javac -cp sikulixapi.jar:. Test01.java}\\\texttt{java -cp sikulixapi.jar:. Test01}\\ Syntaxe, kterou SikuliX v~Java API využívá, je velmi podobná té v~SikuliX-IDE.
	
		\subsection{První test}
		První test s~použitím Java API, viz kód \ref{PrvniJavaAPI}, je téměř identický s~tím, který byl vytvořen pomocí SikuliX-IDE.
		
			\begin{lstjava}{caption={První test Java API}, label={PrvniJavaAPI}}
import org.sikuli.basics.Settings;
import org.sikuli.script.*;
import javax.swing.*;

public class Test01 {

  static Screen s;
  static App prohlizec;
  
  public static void main(String [] args) {
    Settings.OcrTextSearch = true;
    Settings.OcrTextRead = true;

    s= new Screen();
    prohlizec = new App("google-chrome");
    prohlizec.open();
    prohlizec.focus();
    
    try {
      s.wait("obr1.png");
      s.paste("obr2.png");
      s.type(Key.ENTER);
      s.click("obr3.png");
      s.wait("obr4.png");
      s.paste("obr5.png", "1");
      s.click(new Pattern("obr6.png").targetOffset(
        -27,-18));
      s.click("obr7.png");
      s.click("obr4.png");
      String t = s.find("obr8.png").right(
        100).text();
      if (Double.parseDouble(t) == 2.54) {
        JOptionPane.showMessageDialog(null, "Ok" +
          " textove");
      } else {
        JOptionPane.showMessageDialog(null, "Chyba");
      }
      if (s.exists("obr9.png") != null) {
        JOptionPane.showMessageDialog(null, "Ok" +
          " obrazove");
      } else {
        JOptionPane.showMessageDialog(null, "Chyba");
      }
      prohlizec.close();
    } catch (Exception e) {
      e.printStackTrace();
    }
  }
}
\end{lstjava}
	
		\subsection{Sofistikovanější testy}
		S~využitím knihoven \emph{JUnit} a~\emph{Log4j} (ani jedna z~těchto knihoven není pro běh SikuliX bezprostředně nutná) byly vytvořeny čtyři testy, viz kód \ref{DalsiJavaAPI}. První test skončí negativně, druhý pozitivně, třetí pozitivně a~čtvrtý negativně.
		
		Knihovna \emph{JUnit} byla použita z~toho důvodu, že nám pomůže jednak s~organizací testů a~jejich spouštěním, a~jednak s~jejich vyhodnocováním. Dále obsahuje metody pro vyhodnocování a~porovnávání hodnot, tzv. \emph{asserty}.
		
		\emph{Log4J} je knihovna, která slouží k~logování informací do souborů. Umožňuje vlastní konfiguraci výstupních souborů a~spoustu dalších funkcí. Použita byla z~toho důvodu, že během testování je vhodné zaznamenávat prováděné činnosti z důvodu jednoduššího zjištění selhání testu. SikuliX poskytuje informace o~tom, kam klikal či psal. Ty vypisuje na standardní výstup, avšak poskytuje metodu, které se přidá instance \emph{Loggeru}, který poté SikuliX použije pro logování.

\chapter{Sada ukázkových testů a~jejich scénářů}
Existuje webová aplikace \emph{Převodík} dostupná z~url \url{http://oks.kiv.zcu.cz/Prevodnik/Prevodnik} se záměrnými chybami. Pro potřeby práce a~pro názornou ukázku odlišnosti přístupu k~testování webové a~desktopové aplikace byla vytvořena téměř identická aplikace Převodník pomocí technologie JavaFX, jejích kód se nachází na přiloženém CD. Následující sada testů a~jejich scénářů se vztahuje k~těmto aplikacím, jejichž vzhled je ukázán na obrázcích \ref{PrevodnikWeb} a~\ref{PrevodnikJavaFX}.

	\begin{figure}[ht!]
		\centering
		\caption{Webová aplikace Převodník}
		\label{PrevodnikWeb}
		\includegraphics[width=13.5cm]{img/PrevodnikWeb.png}
	\end{figure}
	\begin{figure}[ht!]
		\centering
		\caption{Aplikace Převodník vytvořená pomocí JavaFX}
		\label{PrevodnikJavaFX}
		\includegraphics[width=13.5cm]{img/PrevodnikJavaFX.png}
	\end{figure}

Ze scénářů i~z~přiloženého zdrojového kódu je patrné, že přístup k~testování obou aplikací je totožný. Testy vypadají stejně jak pro desktopovou aplikaci, tak pro webovou. Jediné rozdíly nastávají ve způsobu spouštění aplikací, zacházení s~nimi a~v~cestě k~použitým řídícím screenshotům. Příčinou je to, že se jedná o~téměř identicky vypadající a~chovající se aplikace. Dále budou uvedeny vždy jen stručné ukázky testů s~vyznačenými případnými odlišnostmi.

	\section{Rozdělení testů}
	Scénáře byly rozděleny do tří částí a~každá část poté může obsahovat další skupiny, které sjednocují tematicky si blízké testy. Struktura tedy vypadá přibližně takto:
		{\renewcommand{\labelenumii}{\theenumii}
		\renewcommand{\theenumii}{\theenumi.\arabic{enumii}.}
		\begin{enumerate}
		\item Statické prvky
			\begin{enumerate}
			\item Přítomnost prvků
			\item Editovatelnost polí
			\item Úplnost výběrových seznamů
			\end{enumerate}
		\item Převody
			\begin{enumerate}
			\item Happy Day Scenario
			\item Stejné jednotky
			\item Varianty Vstup OK
			\item Vše na metr
			\item Vše z~metru
			\item Varianty Vstup CHYBA
			\item Všechny vstupy na všechny výstupy
			\item Hraniční hodnoty
			\end{enumerate}
		\item Vymazání
		\end{enumerate}}
		
	Testovací případy mají předponu \emph{TC} za níž nasleduje číslo testovacího případu a~jeho název. Pokud tedy např. testujeme úplnost výstupního výběrového seznamu, název bude \emph{TC01\_03\_02VystupniVyberovySeznam}. Číslo testovacího případu se tvoří dle jeho příslušnosti do části (TestSuite) ve výčtu výše. Patří do \emph{Úplnost výběrových seznamů} a tomu odpovídá první část čísla, \emph{01\_03}. Označení \emph{02} je pořadové číslo testu v~rámci dané kategorie.
		
	\section{Statické prvky}
	Scénáře v~této části pouze zkontrolují, zda testovaná aplikace obsahuje všechny prvky, jako např. tlačítka či vstupní pole. Dále se zkoumá, zda je vstupní pole editovatelné a~výstupní pole nikoli. Poté se zjistí, zda jsou ve výběrových seznamech obsaženy všechny položky.
	
	\section{Převody}
	V~této části jsou zpracované funkční testy konkrétních převodů. Nejprve se provedou testy Happy Day Scenario -- scénář, kdy vše dopadne podle očekávání. Poté se zkontrolují převody mezi stejnými jednotkami, převody s~možnými korektními i~nekorektními vstupy, převody mezi všemi jednotkami a~nakonec převody s~hraničními hodnotami.
	
	\section{Vymazání}
	V~této části se testuje funkčnosti tlačítka Vymazat. Otestuje se případ, kdy se nevyskytla chybová hláška, i~ten, kdy se vyskytla.
	
	\section{Zjištěné chyby}
	Během testování aplikace bylo zjištěno několik chyb. Jak již bylo řečeno dříve, tyto chyby jsou v~aplikaci zaneseny záměrně.
	
		\subsection{Chybné převody z~(na) decimetry}
		Pokud provádíme převod z~(případně na) decimetry, dostaneme nesprávný výsledek, viz \ref{ChybaDm}. Chování odpovídá převodu z~(na) palce.
			\begin{figure}[ht!]
				\centering
				\caption{Chybný převod z decimetru na centimetr}
				\label{ChybaDm}
				\includegraphics[width=13.5cm]{img/Chyby/Dm.png}
			\end{figure}
		
		\subsection{Chybné zaokrouhlení}
		Dále u~jednotek decimetry i~palce v~situaci, kdy jsou použity jak na vstupu, tak na výstupu, je hodnota 3 převedena přibližně na 2.9999996, viz \ref{Zaokrouhleni}.
			\begin{figure}[ht!]
				\centering
				\caption{Chybné zaokrouhlení při převodu mezi decimetry}
				\label{Zaokrouhleni}
				\includegraphics[width=13.5cm]{img/Chyby/Zaokrouhleni.png}
			\end{figure}
		
		\subsection{Převod záporné hodnoty}
		Při zadání záporné hodnoty pro převod se zobrazí chybová hláška o~záporném čísle, avšak převod se i~tak provede, viz \ref{ZapornaHodnota}.
			\begin{figure}[ht!]
				\centering
				\caption{Převod záporné hodnoty}
				\label{ZapornaHodnota}
				\includegraphics[width=13.5cm]{img/Chyby/ZapornaHodnota.png}
			\end{figure}
		
		\subsection{Tlačítko Vymaž}
		Tlačítko \emph{Vymaž} nenastaví všem komponentám výchozí hodnoty. Pouze vymaže obsah vstupního pole. Výstupní pole a~výběrové seznamy nadále obsahují hodnoty z~posledního převodu, viz \ref{Vymazani}.
			\begin{figure}[ht!]
				\centering
				\caption{Použití tlačítka Vymaž}
				\label{Vymazani}
				\includegraphics[width=13.5cm]{img/Chyby/Vymazani.png}
			\end{figure}

\chapter{Problémy}
Během tvorby testů je možné narazit na různé problém. Ty, které byly zjištěny během vytváření této práce, jsou zde uvedeny, popsány a~je k~nim nastíněno možné řešení.

	\section{Rozlišení obrazovky}
	Jelikož se SikuliX v aplikaci orientuje podle screenshotů, je v~danou chvíli závislé na rozlišení, ve kterém byl snímek pořízen. To je z~důvodu, že ještě není implementována funkce, která by měla tento problém odstranit. Pokud se změní rozlišení, nebude daný prvek nalezen, ačkoli bude možné pouhým okem zjistit, že ve skutečnosti přítomen je. Stejný problém nastává i~pokud se změní např. písmo nebo velikost webové stránky s~aplikací a~podobnými změnami vzhledu.
	
	Jedním z~možných řešení je, že si uděláte screenshoty pro různá rozlišení, písma nebo velikosti stránek, případně budete testovat pouze s~jedním daným rozlišením, písmem nebo velikostí stránky.
	
	\section{Rozměry screenshotů}
	Pokud se v~testu hledá prvek v~závisloti na pozici jiného, je možné, že nebude nalezen. Důvodem mohou být rozdílné rozměry screenshotů v~kombinaci s~použitými metodami hledání -- první prvek nalezneme, ale screenshot druhého je větší, než prohledávaná oblast vymezená rozměry prvního, tudíž nemůže být nalezen.
	
	\section{Ukazatel myši}
	Při pořizování screenshotů je vhodné vyvarovat se umístění ukazatele myši ve snímané oblasti. Ten se totiž v~průběhu testu v~této oblasti vyskytovat nemusí a~hledaný prvek by tak nemusel být rozpoznán.
	
	Stejně tak je důležité uvědomit si, že na některých platformách se simulováním pohybu myši a~klikáním mění pozice ukazatele. To může vyústit v~problém, pokud je ukazatel umístěn přes hledaný prvek, který tím pádem pravděpodobně nebude rozpoznán.
	
	\section{Nespolehlivé OCR}
	SikuliX poskytuje možnost rozpoznání textu v~obrázcích. Tato funkcionalita je však v~experimentální fázi a~na jejím vývoji se stále pracuje. Je tedy nespolehlivá a~pro naše účely nevhodná. Text v~obrázku buď vůbec nebyl nalezen (pokud se jednalo např. o jedinou číslici), nebo byl špatně rozpoznán (záměna O a 0, získána pouze část textu, apod.).
	
	Možným řešením je tedy nasimulovat korektní výstup, provést screenshot a ten použít pro obrazové rozpoznání správného výsledku.
	
	\section{Nefunkčnost některých metod, tříd}
	Java API SikuliX poskytuje třídy a metody pro práci s aplikacemi a jejich okny. Tyto metody a třídy mají však občas jiné než očekávané chování. Vzhledem k téměř nulové dokumentaci je toto celkem velký problém. Konkrétním příkladem je např. to, pokud bychom chtěli testování omezit pouze na okno aplikace. SikuliX je schopno okno s aplikací najít podle (části) jejího titulku a přenést jej do popředí. Už ale není schopno získat rozměry a pozici tohoto okna, ačkoli metody pro tyto funkce existují.
	
	Postup, kterým se dá tato funkcionalita nahradit, je následující. Aplikaci najdeme podle jejího titulku a necháme ji přenést do popředí. Poté je SikuliX schopné získat pozici a rozměry okna, které je v popředí (má tzv. focus).
	
	Dále nedokázalo indikovat, že aplikace ukončila svůj běh. Pokud jsme tedy použili cyklus "testuj, dokud aplikace běží", testování pokračovalo i v případě, že byla aplikace již zavřena.
	
	Náhradním řešením tedy bylo vytvořit screenshot některé části aplikace, která se nemění a je vždy v aplikaci přítomna. Cyklus poté vypadá takto "testuj, dokud najdeš tuto část aplikace". To však není řešení absolutní, protože nebude funkční v případech, kdy se objeví např. dialogové okno, které tuto části zakryje, nebo pokud taková část vůbec neexistuje.
		
\chapter{Závěr}
V~práci jsou popsány některé metody testování grafického uživatelského rozhraní a~také základní důvody používání těchto metod. V~první části práce je proveden kvalifikovaný průzkum existujících a~dostupných nástrojů, které je možné k~testování GUI využít.

Byly vybrány tři nástroje, jejichž stručná charakteristika je uvedena v~teoretické části práce. Dále bylo navrženo multikriteriální hodnocení a~provedeno podrobné porovnání zvolených nástrojů. Multikriteriální hodnocení má sedm kritérií. Jejich váhy si může zájemce určit sám, viz tabulkový soubor na CD. Podle vah diskutovaných s~vedoucím práce byl vybrán nástroj SikuliX, který byl dále detailně zkoumán.

Možnosti nástroje SikuliX byly okamžitě ověřovány v~připravovaných praktických testech. Aby bylo možné demonstrovat, že SikuliX má širší možnosti pro testování než nástroj pro testování webových aplikaci Selenium, bylo nutné nejdříve připravit identickou desktopovou aplikaci k~již existující webové aplikaci. Podle již existujícího vzoru byly připraveny v~SikuliX dvě kompletní sady testů jak pro desktopovou, tak pro webovou aplikaci. Tento přístup dovoluje porovnat oba dva způsoby testování a~poukázat na případné drobné rozdíly mezi nimi.

Při testování byly využity možnosti frameworku JUnit a~rovněž předpřipravené možnosti SikuliX pro logování, které nebývá v~testování tak běžné. Zde se ale ukazuje, že je výhodné.

Pozornost byla v~práci věnována i~monkey testům a~automatizaci činností, přičemž bylo zjištěno, že obě dvě aktivity se nijak výrazně neodlišují od psaní běžných testů.

V~práci je též stručně porovnána časová náročnost nástrojů, ale je třeba říci, že toto není podstatné kritérium výběru nástroje.

V~této práci bylo vytvořeno dvakrát 103 testů (každý pro webovou a~JavaFX aplikaci, dohromady tedy 206 testů). Počet snímků potřebných pro vytvoření testů byl čtyřikrát 74 (každý pro webovou a~JavaFX aplikaci jak na Linux, tak na Windows, celkem tedy 296 snímků). Dále byl vytvořen jeden monkey test a~jeden skript pro ukázku automatizace činnosti. Dohromady bylo vytvořeno více než 330 souborů a~napsáno více než 7350 řádek zdrojového kódu.

Výsledky práce budou přímo použity pro přípravu jedné z~přednášek předmětu Ověřování kvality softwaru vyučovaného na KIV.
 
% 
% PRO ANGLICKOU SAZBU JE NUTNÉ ZMĚNIT
% CITAČNÍ STYL!
%
\bibliographystyle{csplainnatkiv}
{\raggedright\small
\bibliography{literatura}
}

\chapter*{Příloha A}
\addcontentsline{toc}{chapter}{Příloha A}
	\begin{lstpython}{caption={První skript}, label={PrvniSkript}}
prohlizec = App("google-chrome")
prohlizec.open()	#otevre aplikaci definovanou
				#vyse
prohlizec.focus()	#vybere do popredi jeji okno
#ceka, dokud na obrazovce nenajde obrazek
wait("obr1.png")
#najde na obrazovce obrazek a vlozi do neho text
paste("obr2.png", "http://oks.kiv.zcu.cz/Prevodnik")
type(Key.ENTER)	#simuluje stisk klavesy ENTER
#najde na obrazovce obrazek a klikne na nej
click("obr3.png")
wait("obr4.png")
paste("obr5.png", "1")
#klikne o 27px vyse a 18px vlevo od nalezeneho
#obrazku
click(Pattern("obr6.png").targetOffset(-27,-18))
click("obr7.png")
click("obr4.png")
#precte text z casti, ktera je 100px vpravo od
#nalezeneho obrazku
T = find("obr8.png").right(100).text()
if T == "2.54":
	#pokud rozpoznany text souhlasi se zadanym,
	#otevre se vyskakovaci okno
    popup("Ok textove")
else:
    popError("Chyba")    #jinak se zobrazi chybove
    			 #okno

if exists("obr9.png"):
	#pokud na obrazovce existuje obrazek, otevre
	#se vyskakovaci okno
    popup("Ok obrazove")
else:
    popError("Chyba")
prohlizec.close()    #ukonci aplikaci
	\end{lstpython}

	\begin{lstjava}{caption={První test Java API}, label={PrvniJavaAPI}}
import org.sikuli.basics.Settings;
import org.sikuli.script.*;
import javax.swing.*;

public class Test01 {

  static Screen s;
  static App prohlizec;
  
  public static void main(String [] args) {
    Settings.OcrTextSearch = true;
    Settings.OcrTextRead = true;

    s= new Screen();
    prohlizec = new App("google-chrome");
    prohlizec.open();
    prohlizec.focus();
    
    try {
      s.wait("obr1.png");
      s.paste("obr2.png");
      s.type(Key.ENTER);
      s.click("obr3.png");
      s.wait("obr4.png");
      s.paste("obr5.png", "1");
      s.click(new Pattern("obr6.png").targetOffset(
        -27,-18));
      s.click("obr7.png");
      s.click("obr4.png");
      String t = s.find("obr8.png").right(
        100).text();
      if (Double.parseDouble(t) == 2.54) {
        JOptionPane.showMessageDialog(null, "Ok" +
          " textove");
      } else {
        JOptionPane.showMessageDialog(null, "Chyba");
      }
      if (s.exists("obr9.png") != null) {
        JOptionPane.showMessageDialog(null, "Ok" +
          " obrazove");
      } else {
        JOptionPane.showMessageDialog(null, "Chyba");
      }
      prohlizec.close();
    } catch (Exception e) {
      e.printStackTrace();
    }
  }
}
	\end{lstjava}
	
	\begin{lstjava}{caption={Další testy Java API}, label={DalsiJavaAPI}}
import org.apache.logging.log4j.LogManager;
import org.apache.logging.log4j.Logger;
import org.junit.AfterClass;
import org.junit.Before;
import org.junit.BeforeClass;
import org.junit.Test;
import org.junit.rules.ErrorCollector;
import org.sikuli.basics.Debug;
import org.sikuli.basics.Settings;
import org.sikuli.script.*;
import javax.swing.*;
import java.time.LocalDateTime;
import static org.junit.Assert.*;

public class Test01 {

  static Logger logger;
  static ErrorCollector collector;
  static Screen s;
  static App prohlizec;
  static boolean run;

  static {
    System.setProperty("log4j.configurationFile",
      "log-konfigurace.xml");
  }

  private String nazevScreenshotu() {
    LocalDateTime l = LocalDateTime.now();
      return l.getYear() + "" + l.getMonthValue() +
        "" + l.getDayOfMonth() + "" + l.getHour() +
        "" + (l.getMinute() < 10 ? "0" + l.
        getMinute() : l.getMinute()) + "" + l.
        getSecond() + "";
  }

  @BeforeClass
  public static void setUpBeforeClass() {
    logger = LogManager.getLogger();

    Settings.OcrTextSearch = true;
    Settings.OcrTextRead = true;
    Debug.setLogger(logger);
    Debug.setLoggerAll("info");

    collector = new ErrorCollector();
    s = new Screen();
    prohlizec = new App("google-chrome");
    prohlizec.open();
    prohlizec.focus();
    run = true;
    }

  @AfterClass
  public static void tearDownAfterClass() {
    JOptionPane.showMessageDialog(null, "Script" +
      " dokoncen");
    prohlizec.close();
  }

  @Before
  public void setUp() {
    try {
      s.wait("png/addressBar.png", 10);
      s.click(new Pattern("png/addressBar.png").
        targetOffset(100, 0));
      s.paste("http://oks.kiv.zcu.cz/Prevodnik");
      s.type(Key.ENTER);
      s.wait("png/zalozkaPrevodnik.png", 5);
    } catch (Exception e) {
      run = false;
      s.capture().save("errors", nazevScreenshotu());
      logger.error(e.getMessage());
    }
  }

  @Test
  public void testPorovnejText() {
    if (run) {
      try {
        s.click("png/zalozkaPrevodnik.png");
        s.wait("png/tlacitkoPreved.png", 5);
        s.paste("png/vstup.png", "1");
        Match m = s.find("png/jednotky.png");
        m.setTargetOffset(-27, -18);
        m.click();
        s.findText("(metr)").click();
        s.click(new Pattern("png/jednotky.png").
          targetOffset(-27, 18));
        s.find("png/dm.png").click();
        s.click("png/tlacitkoPreved.png");
        String t = s.find("png/vystup.png").right(
          100).text();
        assertEquals(10, Double.parseDouble(t),
          0.01);
      } catch (FindFailed | AssertionError e) {
        s.capture().save("errors",
          nazevScreenshotu());
        logger.error(e.getMessage());
        fail(e.getMessage());
      }
    } else {
      run = true;
      logger.error("setUp neuspesny");
      fail("setUp neuspesny");
    }
  }

  @Test
  public void testPorovnejObraz() {
    if (run) {
      try {
        s.click("png/zalozkaPrevodnik.png");
        s.wait("png/tlacitkoPreved.png", 5);
        s.paste("png/vstup.png", "1");
        s.click(new Pattern("png/jednotky.png").
          targetOffset(-27, -18));
        s.click("png/inch.png");
        s.click("png/tlacitkoPreved.png");
        assertTrue(s.exists("png/vysledek.png") !=
          null);
      } catch (FindFailed | AssertionError e) {
        s.capture().save("errors",
          nazevScreenshotu());
        logger.error(e.getMessage());
        fail(e.getMessage());
      }
    } else {
      run = true;
      logger.error("setUp neuspesny");
      fail("setUp neuspesny");
    }
  }

  @Test
  public void testZkontrolujOdkazObrazekKiv() {
    if (run) {
      try {
        s.click("png/logoKiv.png");
        assertTrue(s.exists("png/zahlaviKiv.png") !=
          null);
      } catch (FindFailed | AssertionError e) {
        s.capture().save("errors",
          nazevScreenshotu());
        logger.error(e.getMessage());
        fail(e.getMessage());
      }
    } else {
      run = true;
      logger.error("setUp neuspesny");
      fail("setUp neuspesny");
    }
  }

  @Test
  public void testChyba() {
    if (run) {
      try {
        s.wait("png/tlacitkoPreved.png", 5);
        s.paste("png/vstup.png", "1");
        s.click(new Pattern("png/jednotky.png").
          targetOffset(-27, -18));
        s.findText("(metr)").click();
        s.click("png/tlacitkoPreved.png");
        String t = s.find("png/vystup.png").right(
          100).text();
        assertEquals(100, Double.parseDouble(t),
          0.01);
      } catch (FindFailed | AssertionError e) {
        s.capture().save("errors",
          nazevScreenshotu());
        logger.error(e.getMessage());
        fail(e.getMessage());
      }
    } else {
      run = true;
      logger.error("setUp neuspesny");
      fail("setUp neuspesny");
    }
  }
}
	\end{lstjava}
\end{document}
